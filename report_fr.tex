\documentclass[12pt]{article}

\setlength{\parskip}{1em}

\usepackage[T1]{fontenc}
\usepackage[french]{babel}
\usepackage{enumitem}
\usepackage{textcomp}
\usepackage{amsfonts}
\usepackage{slantsc}
\usepackage{graphicx}
\usepackage{amsmath}
\usepackage[a4paper, margin=1in]{geometry}

\usepackage{hyperref}
\hypersetup{
  colorlinks=true,
  linkcolor=blue,
  urlcolor=blue,
  pdftitle={Projet LO21: Rapport final}
}

\usepackage{listings}
\lstset{
  numbers=left,
  columns=fullflexible,
  language=C,
  numberstyle=\scriptsize,
}

\usepackage[linesnumbered, french]{algorithm2e}
\SetKwInput{Data}{Donn\'ees}
\SetKwInput{Result}{R\'esultat}
\SetKwInput{Vars}{Variables}
\SetKwInput{Assertion}{Assertion}
\SetKwInput{KWClass}{Classe}
\newcommand{\Class}[1]{\KWClass{$\mathcal{O}(#1)$}}
\SetKwIF{If}{ElseIf}{Else}{Si}{alors}{Sinon si}{Sinon}{FinSi}
\SetKwFor{While}{Tant que}{faire}{Fin TantQue}
\SetKwBlock{Begin}{D\'ebut}{Fin}
\SetKwRepeat{Repeat}{Faire}{Tant que}
\DontPrintSemicolon
\SetAlgoLined

\RestyleAlgo{boxruled}

\newcommand{\Assign}[2]{#1 $\; \longleftarrow \;$ #2}
\newcommand{\Arg}[2]{\hspace{0.2em}#1\hspace{0.05em}: #2}
% \newcommand{\Child}[2]{(#1 $\rightarrow$ #2)}
\newcommand{\Child}[2]{#2(#1)}
\newcommand{\Null}[0]{\textsc{null}\hspace{4pt}}
\newcommand{\Et}[2]{#1\hspace{2pt}\textnormal{\textbf{et}}\hspace{2.5pt}#2}
\newcommand{\Not}[1]{\textnormal{\textbf{non}}(#1)}
% \newcommand{\Neither}[2]{\Et{\Not{#1}}{\Not{#2}}}

\renewcommand{\arraystretch}{1.5}

\title{Projet LO21: Rapport final}
\author{Adrien Burgun}
\date{Automne 2020}
\graphicspath{{report/}}

\begin{document}

\maketitle

\begin{abstract}

  Le projet de ce semestre pour le cours de \textbf{LO21} (Algorithmique et Programmation II) porte sur un \textit{\og système expert \fg}.
  Un système expert est constitué de 3 éléments:

  \begin{description}[align=left]
    \item [Une base de connaissance,] qui prend la forme suivante:
    \[
      A \land B \land ... \land Z \Rightarrow \Omega
    \]
    Où \(A, B, ...\) sont les symboles (d'arité zéro, aussi appelés \og propositions \fg) constituant la \textit{prémisse} et \(\Omega\) est la \textit{conclusion}.

    \item [Une base de faits,] qui est la liste des symboles ayant la valeur \textit{\og Vrai \fg} (qui correspond à l'état \textit{\og Certain \fg}). \\
    Un symbole ne faisant pas partie de cette liste a par défaut la valeur \textit{\og Faux \fg} (qui correspond à l'état \textit{\og Incertain \fg}).

    \item [Un moteur d'inférence,] qui, à partir de la base de connaissance et la base de faits, déduit quels autres symboles sont aussi vrais et les ajoute à la base de faits.
  \end{description}

  Nous définirons d'abords le type \textit{\og Règle \fg}, constituant la base de connaissance. \\
  Nous définirons ensuite le type \textit{\og BC \fg} (\underline{B}ase de \underline{C}onnaissance). \\
  Nous décrirons enfin le moteur d'inférence comme implémenté dans ce projet, avec différents exemples.
\end{abstract}

\newpage

% 1
\section{Règles}

Soit \textbf{Règle} le type représentant une règle sous la forme d'une liste de symboles:

% Faire un tableau?

% \begin{lstlisting}
% #define LONGUEUR_SYMBOLE 256

% struct regle {
%   char symbole[LONGUEUR_SYMBOLE];
%   struct regle* suivant;
% };

% typedef struct regle regle_t;
% \end{lstlisting}

\begin{tabular}{|p{3cm}|p{4cm}|p{6.5cm}|}
  \hline
  \multicolumn{3}{|c|}{\textbf{Structure 1 :} Règle\label{R}} \\
  \hline
  \textbf{Nom} & \textbf{Type} & \textbf{Description} \\
  \hline
  \textit{symbole} & Règle $\rightarrow$ Symbole & Retourne le nom du symbole correspondant au noeud en tête de liste. \\
  \hline
  \textit{suivant} & Règle $\rightarrow$ Règle & Retourne une référence au prochain élément de la liste, \textit{règle\_vide} si l'élément est le dernier de la liste. \\
  \hline
  \textit{nouvelle\_règle} & ((Symbole) $\times$ Règle) $\rightarrow$ Règle & Compose une nouvelle règle à partir du nom d'un symbole et une référence à la prochaine règle. \\
  \hline
  \textit{règle\_vide} & Règle & La règle vide. \\
  \hline
  \textit{mettre\_suivant} & (Règle $\times$ Règle) $\rightarrow$ Règle & Modifie une Règle pour y attacher une Règle comme règle suivante; retourne également la règle modifiée. \\
  \hline
\end{tabular}

Le type \og Symbole \fg correspond dans l'implémentation C à une chaîne de caractères.
Le dernier élément d'une telle liste correspond à la conclusion de la règle, tandis que tous les autres éléments appartiennent à la prémisse (contrainte du projet).

Les axiomes sur ces fonctions sont:

\begin{itemize}
\item \textit{symbole}(\textit{nouvelle\_règle}(s, r)) = s
\item \textit{suivant}(\textit{nouvelle\_règle}(s, r)) = r
\item \Assign{A}{\textit{nouvelle\_règle}(s, r)}; \textit{mettre\_suivant}(a, r') $\; \Rightarrow$ \textit{suivant}(A) = r'
\end{itemize}

Nous pouvons noter que:

Liste $\prec$ Règle [$\emptyset$: \textit{règle\_vide}, tête: \textit{symbole}, reste: \textit{suivant}, insérer\_tête: \textit{nouvelle\_règle}, Élement: Symbole]

(Soit également un type concret implémentant Règle implémente également Liste(Symbole))

Nous avons implémenté ce type abstrait de donnée sous forme d'une liste chaînée, de part sa facilité d'implémentation et la difficulté d'implémenter un type abstrait de donnée grandissable par la tête sous tout autre forme qu'une liste chaînée.

Le fait que ce type abstrait soit grandissable par la tête reflète également la difficulté de décrire de manière concise et simple à comprendre toute autre forme de type abstrait représentant une liste.

% 1.1
\subsection{Créer une règle vide}

Nous représentons une règle vide par \textit{règle\_vide}.
Voici l'algorithme permettant de créer une règle vide:

\begin{algorithm}[H]
\Vars{\Arg{$R$}{La règle vide à retourner}}
\Result{\Arg{$R$}{Règle}}
\Class{1}

\Begin({RègleVide()}){
  \Assign{$R$}{\textit{règle\_vide}}
}

\caption{RègleVide\label{RV}}
\end{algorithm}

% 1.2
\subsection{Ajouter une proposition à la prémisse d'une règle}

% Je me permets d'interrompre votre brave lecture de ce document latex pour argumenter sur la question posée dans le sujet original, qui stipule que cet ajout doit se faire en *queue*, et non pas en *tête*.
% Voici donc des exemples d'utilisations de cette fonction et les résultats auxquels on s'attendrait (1) par rapport aux résultats qu'on obtiendra avec cette contrainte (2):
%
% ajout_prémisse(ø, A) -> (1) A | ø, (2) ø | A
% ajout_prémisse(ø | A, B) -> (1) B | A; (2) A | B
% ajout_prémisse(A & B | Ω, C) -> (1) C & A & B | Ω; (2) A & B & Ω | C
%
% Le problème est, à mon avis, qu'ajouter un symbole à la prémisse d'une règle ne devrait pas affecter la conclusion d'une règle, ce qui n'est pas le cas lorsque l'ajout se fait à l'endroit dédié à la conclusion de la règle.
% Cela voudra aussi dire que la fonction pour ajouter une conclusion à une règle sera identique à celle-ci.
%
% PS: après plusieures semaines de travail sur ce projet, je pense que le sujet attendait que l'on utilise une liste grandissable en queue pour Règle, et que l'on décrive ses fonctions de manière sémantique, et non de manière axiomatique.

L'ajout des propositions (symboles) à la prémisse d'une règle se fait par l'algorithme \textit{AjoutPrémisse} défini ci-dessous.
Cet ajout se fait en queue de la liste (contrainte du projet).

La liste donnée en entrée est modifiée par l'algorithme et est ensuite retournée; ceci est dû aux contraintes des listes chaînées dans l'implémentation C et du type abstrait de donnée Règle: si la liste est initialement vide (représenté par \Null et \textit{règle\_vide}), alors nous ne pouvons pas muter celle-ci en utilisant \textit{mettre\_suivant}.
Ceci est géré par le premier \textbf{Si}.

\newpage

\begin{algorithm}[H]
\Vars{
  \begin{itemize}
    \item \Arg{$R$}{La règle à modifier}
    \item \Arg{$R'$}{Une variable temporaire pour traverser la liste}
    \item \Arg{\textit{symbole}}{Le nom de la proposition (symbole) à insérer}
  \end{itemize}
}
\Data{\Arg{$R$}{Règle}, \Arg{\textit{symbole}}{Symbole}}
\Result{\Arg{$R$}{Règle}}
\Assertion{$R$ n'a pas encore de conclusion}
\Class{n}

\Begin({AjoutPrémisse($R$, \textit{symbole})}){
  \uIf{$R$ = \textit{règle\_vide}}{
    \Assign{$R$}{\textit{nouvelle\_règle}(\textit{symbole}, \textit{règle\_vide})}
  }
  \Else{
    \Assign{$R'$}{R}

    \tcp{Répeter jusqu'à ce qu'on atteigne le dernier élément}

    \While{\Child{$R'$}{\textit{suivant}} $\neq$ \textit{règle\_vide}}{
      \Assign{$R'$}{\Child{$R'$}{\textit{suivant}}}
    }

    \tcp{$R'$ contient désormais le dernier élément de la liste}

    \textit{mettre\_suivant}($R'$, \textit{nouvelle\_règle(\textit{symbole}, \textit{règle\_vide})})
  }
}

\caption{AjoutPrémisse\label{AP}}
\end{algorithm}

% 1.3
\subsection{Créer la conclusion d'une règle}

Créer la conclusion d'une règle revient à ajouter une proposition (symbole) à la fin de la règle.
Pour ce faire, nous ré-utilisons l'algorithme \hyperref[AP]{AjoutPrémisse} défini plus tôt.

\begin{algorithm}[H]
\Vars{
  \begin{itemize}
    \item \Arg{$R$}{La règle à modifier}
    \item \Arg{\textit{symbole}}{Le nom de la proposition (symbole) à insérer comme conclusion}
  \end{itemize}
}
\Data{\Arg{$R$}{Règle}, \Arg{\textit{symbole}}{Symbole}}
\Result{\Arg{$R$}{Règle}}
\Assertion{$R$ n'a pas encore de conclusion}
\Class{n}

\Begin({AjoutConclusion($R$, \textit{symbole})}){
  \Assign{$R$}{\hyperref[AP]{AjoutPrémisse}($R$, \textit{symbole})}
}

\caption{AjoutConclusion\label{AC}}
\end{algorithm}

% 1.4
\subsection{Tester si une proposition appartient à la prémisse d'une règle}

Nous testons si une proposition appartient à la prémisse d'une règle en traversant celle-ci de manière récursive (contrainte du projet).

Les 3 cas minimaux sont:

\begin{description}
  \item[$R$ = {[]}] (règle vide): retourner \og Faux \fg
  \item[$R$ = \{symbole: "...", suivant: \textit{règle\_vide}\}] (conclusion): retourner \og Faux \fg
  \item[$R$ = \{symbole: symbole\_recherché, suivant: ...\}] (symbole trouvé): retourner \og Vrai \fg
\end{description}

Dans les autres cas, nous retournons de manière récursive le résultat de la même fonction, appelée sur \Child{$R$}{\textit{suivant}}.

\begin{algorithm}[H]
\Vars{
  \begin{itemize}
    \item \Arg{$R$}{La règle à étudier}
    \item \Arg{\textit{symbole}}{La nom de la proposition à rechercher}
    \item \Arg{\textit{résultat}}{Si oui ou non la proposition à été trouvée}
  \end{itemize}
}
\Data{\Arg{$R$}{Règle}, \Arg{\textit{symbole}}{Symbole}}
\Result{\Arg{\textit{résultat}}{Booléen}}
\Assertion{$R$ a une conclusion}
\Class{n}

\Begin({TestAppartenance($R$, \textit{symbole})}){
  \uIf{$R$ = \textit{règle\_vide}}{
    \tcp{Règle vide}
    \Assign{\textit{résultat}}{Faux}
  }
  \uElseIf{\Child{$R$}{\textit{suivant}} = \textit{règle\_vide}}{
    \tcp{Conclusion}
    \Assign{\textit{résultat}}{Faux}
  }
  \uElseIf{\Child{$R$}{\textit{symbole}} = \textit{symbole}}{
    \tcp{Symbole trouvé}
    \Assign{\textit{résultat}}{Vrai}
  }
  \Else{
    \Assign{\textit{résultat}}{TestAppartenance(\Child{$R$}{\textit{suivant}}, \textit{symbole})}
  }
}

\caption{TestAppartenance\label{TA}}
\end{algorithm}

% 1.5
\subsection{Supprimer une proposition de la prémisse d'une règle}

Nous supprimons une proposition de la prémisse d'une règle de manière récursive.
Cette décision est motivée par le format de Règle et sa simplicité d'implémentation.
Les cas minimaux sont les suivants:

\begin{description}
  \item[$R$ = {[]}] (règle vide): retourner \textit{règle\_vide}
  \item[$R$ = \{symbole: "...", suivant: \textit{règle\_vide}\}] (conclusion): retourner $R$
\end{description}

Dans le cas général, nous attribuons à \Child{$R$}{\textit{suivant}} la valeur retournée par cette fonction, appelée sur \Child{$R$}{\textit{suivant}}, puis nous retournons soit \Child{$R$}{\textit{suivant}} si le noeud correspond au symbole, soit $R$ sinon.

\begin{algorithm}[H]
\Vars{
  \begin{itemize}
    \item \Arg{$R$}{La règle à modifier}
    \item \Arg{\textit{symbole}}{La nom de la proposition à rechercher}
    \item \Arg{$R'$}{La règle privée de \textit{symbole} dans sa prémisse}
  \end{itemize}
}
\Data{\Arg{$R$}{Règle}, \Arg{\textit{symbole}}{Symbole}}
\Result{\Arg{$R'$}{Règle}}
\Assertion{$R$ a une conclusion}
\Class{n}

\Begin({SupprimerSymbole($R$, \textit{symbole})}){
  \uIf{$R$ = \textit{règle\_vide}}{
    \tcp{Règle vide}
    \Assign{$R'$}{\textit{règle\_vide}}
  }
  \uElseIf{\Child{$R$}{\textit{suivant}} = \textit{règle\_vide}}{
    \tcp{Conclusion}
    \Assign{$R'$}{$R$}
  }
  \Else{
    \uIf{\Child{$R$}{\textit{symbole}} = \textit{symbole}}{
      \tcp{Retourner le reste de la liste, sans ce noeud}
      \Assign{$R'$}{SupprimerSymbole(\Child{$R$}{\textit{suivant}}, \textit{symbole})}
    }
    \Else{
      \Assign{$R'$}{\textit{mettre\_suivant}($R$, SupprimerSymbole(\Child{$R$}{\textit{suivant}}, \textit{symbole}))}
    }
  }
}

\caption{SupprimerSymbole\label{SS}}
\end{algorithm}

% 1.6
\subsection{Tester si la prémisse d'une règle est vide}

Voici la fonction retournant \og Vrai \fg si la prémisse d'une règle est vide et \og Faux \fg si la prémisse d'une règle contient au moins 1 symbole:

\begin{algorithm}[H]
\Vars{
  \begin{itemize}
    \item \Arg{$R$}{La règle à étudier}
    \item \Arg{\textit{résultat}}{Si oui ou non la prémisse d'une règle est vide}
  \end{itemize}
}
\Data{\Arg{$R$}{Règle}}
\Result{\Arg{\textit{résultat}}{Booléen}}
\Assertion{$R$ a une conclusion ou $R$ est vide}
\Class{1}

\Begin({PrémisseVide($R$)}){
  \uIf{$R$ = \textit{règle\_vide}}{
    \tcp{La règle est vide, donc sa prémisse est vide}
    \Assign{\textit{résultat}}{Vrai}
  }
  \uElseIf{\Child{$R$}{\textit{suivant}} = \textit{règle\_vide}}{
    \tcp{La règle n'a qu'une conclusion, donc sa prémisse est vide}
    \Assign{\textit{résultat}}{Vrai}
  }
  \Else{
    \Assign{\textit{résultat}}{Faux}
  }
}

\caption{PrémisseVide\label{PV}}
\end{algorithm}

% 1.7
\subsection{Accéder à la proposition se trouvant en tête d'une prémisse}

Voici la fonction retournant la valeur de la proposition se trouvant en tête d'une prémisse.
Si la prémisse est vide, alors la fonction retourne \textit{symbole\_vide} ($\emptyset$).

\begin{algorithm}[H]
\Vars{
  \begin{itemize}
    \item \Arg{$R$}{La règle à étudier}
    \item \Arg{\textit{résultat}}{La valeur du premier symbole de la prémisse, si existant}
  \end{itemize}
}
\Data{\Arg{$R$}{Règle}}
\Result{\Arg{\textit{résultat}}{Symbole}}
\Assertion{$R$ a une conclusion}
\Class{1}

\Begin({TêteRègle($R$)}){
  \uIf{\hyperref[PV]{PrémisseVide}($R$)}{
    \tcp{La prémisse est vide: nous retournons \textit{symbole\_vide}}
    \Assign{\textit{résultat}}{\textit{symbole\_vide}}
  }
  \Else{
    \Assign{\textit{résultat}}{\Child{$R$}{\textit{symbole}}}
  }
}

\caption{TêteRègle\label{TR}}
\end{algorithm}

% 1.8
\subsection{Accéder à la conclusion d'une règle}

La conclusion se trouvant à la fin d'une règle, nous traversons simplement la liste jusqu'au dernier élément de celle-ci et retournons sa valeur.
Si la liste est vide, alors la fonction retourne \textit{règle\_vide}.

\begin{algorithm}[H]
\Vars{
  \begin{itemize}
    \item \Arg{$R$}{La règle à étudier}
    \item \Arg{$R'$}{Variable temporaire pour traverses la liste}
    \item \Arg{\textit{résultat}}{La valeur du premier symbole de la prémisse, si existant}
  \end{itemize}
}
\Data{\Arg{$R$}{Règle}}
\Result{\Arg{\textit{résultat}}{Symbole}}
\Assertion{$R$ a une conclusion}
\Class{n}

\Begin({ConclusionRègle($R$)}){
  \uIf{\Child{$R$}{\textit{suivant}} = \textit{règle\_vide}}{
    \Assign{\textit{résultat}}{\textit{règle\_vide}}
  }
  \Else{
    \Assign{$R'$}{$R$}

    \tcp{Avançer jusqu'à la fin de la liste}
    \While{\Child{$R'$}{\textit{suivant}} $\neq$ \textit{règle\_vide}}{
      \Assign{$R'$}{\Child{$R'$}{\textit{suivant}}}
    }

    \Assign{\textit{résultat}}{\Child{$R'$}{\textit{symbole}}}
  }
}

\caption{ConclusionRègle\label{CR}}
\end{algorithm}

% 2
\section{Base de Connaissance}

Soit \textbf{BC} le type représentant une base de connaissance (liste de règle); celle-ci prend la forme d'une liste de Règles:

\begin{tabular}{|p{3cm}|p{4cm}|p{6.5cm}|}
  \hline
  \multicolumn{3}{|c|}{\textbf{Structure 2 :} BC\label{BC}} \\
  \hline
  \textbf{Nom} & \textbf{Type} & \textbf{Description} \\
  \hline
  \textit{règle} & BC $\rightarrow$ Règle & Retourne une référence à la règle correspondant à ce noeud. \\
  \hline
  \textit{suivant} & BC $\rightarrow$ BC & Une référence au prochain élément de la liste, \Null si l'élément est le dernier de la liste. \\
  \hline
  \textit{nouvelle\_base} & (Règle $\times$ BC) $\rightarrow$ BC & Crée une nouvelle base à partir d'une référence vers une Règle et d'une référence vers BC \\
  \hline
  \textit{base\_vide} & BC & La base vide \\
  \hline
\end{tabular}

Les axiomes sur ces fonctions sont:

\begin{itemize}
  \item \textit{règle}(\textit{nouvelle\_base}(r, b)) = r
  \item \textit{suivant}(\textit{nouvelle\_base}(r, b)) = b
\end{itemize}

Pour les même raisons que celles de Règle, le type abstrait BC est grandissable par la tête et son implémentation proposée se fait sous la forme d'une liste chaînée.

% 2.1
\subsection{Créer une base vide}

Nous représentons une base vide par \textit{base\_vide}.
Voici l'algorithme permettant de créer une base vide:

\begin{algorithm}[H]
\Vars{\Arg{$B$}{La base vide à retourner}}
\Result{\Arg{$B$}{BC}}
\Class{1}

\Begin({BaseVide()}){
  \Assign{$B$}{\textit{base\_vide}}
}

\caption{BaseVide\label{BV}}
\end{algorithm}

% 2.2
\subsection{Ajouter une règle à une base de connaissance}

L'ajout de règle à la base de connaissance se fait en tête (pour sa simplicité d'implémentation). Voici son algorithme:

\begin{algorithm}[H]
\Vars{
  \begin{itemize}
    \item \Arg{$B$}{La base de connaissance à modifier}
    \item \Arg{\textit{règle}}{La valeur de \textit{règle} à mettre dans le noeud}
    \item \Arg{$B'$}{La base de connaissance contenant la nouvelle règle}
  \end{itemize}
}
\Data{\Arg{$B$}{BC}; \Arg{\textit{règle}}{Règle}}
\Result{\Arg{$B'$}{BC}}
\Class{1}

\Begin({AjoutRègle($B$, \textit{règle})}){
  \Assign{$B'$}{\textit{nouvelle\_base}(\textit{règle}, $B$)}
}

\caption{AjoutRègle\label{AR}}
\end{algorithm}

% 2.3
\subsection{Accéder à la règle se trouvant en tête de la base}

Voici l'algorithme permettant d'accéder à la règle se trouvant en tête; s'il n'y a pas de règle en tête, nous retournons \textit{règle\_vide}.

\begin{algorithm}[H]
\Vars{
  \begin{itemize}
    \item \Arg{$B$}{La base de connaissance à modifier}
    \item \Arg{$R$}{La règle se trouvant en tête de la base, \textit{règle\_vide} si la base est vide}
  \end{itemize}
}
\Data{\Arg{$B$}{BC}}
\Result{\Arg{$R$}{Règle}}
\Class{1}

\Begin({TêteBase($B$)}){
  \uIf{$B$ = \textit{base\_vide}}{
    \Assign{$R$}{\textit{règle\_vide}}
  }
  \Else{
    \Assign{$R$}{\textit{règle}($B$)}
  }
}

\caption{TêteBase\label{TB}}
\end{algorithm}

\section{Moteur d'inférence}

Le moteur d'inférence est un algorithme permettant de déduire à partir de la liste initiale des symboles (propositions) ayant la valeur \og Vrai \fg et de la base de connaissance (BC) la liste de tous les symboles étant vrais.

L'algorithme éxecute un maximum de $n$ (la longueur de la base de connaissance) passages sur les règles, celles-cis prenant la forme suivante ($p \geq 0$ symboles $S_x$ dans la prémisse et le symbole $\Omega_k$ dans la conclusion):

\[BC \vdash (\bigwedge_{x=0}^{x < p} S_x) \Rightarrow \Omega_k\]

\[\text{Avec: } \quad (\bigwedge_{x=0}^{x < p} S_x) = \begin{cases}
  S_0 \land S_1 \land ... \land S_{p-1} & \text{Si } p > 0 \\
  Vrai & \text{Si } p = 0
\end{cases}
\]

Si $\Omega_k$ n'appartient pas encore à la liste des symboles vrais et que $\forall x \in [0, p[, S_x = Vrai$ ou $p = 0$, alors on ajoute $\Omega_k$ à la liste des symboles vrais.

Nous notons par la suite $n$ pour la longueur de la base de connaissance, $p$ pour la longueur maximale des règles de la base de connaissance et $q$ pour le nombre de symboles différents faisant partie de la base de connaissance ou étant initiallement présents dans $S$.
Nous assumons que les symboles ont une longueur maximale fixée au préalable, faisant de leur comparaison une fonction de classe $\mathcal{O}(1)$.

\begin{algorithm}
\Vars{
  \begin{itemize}
    \item \Arg{$B$}{La base de connaissance, de longueur $n$ et dont les règles ont pour longueur maximale $p$.}
    \item \Arg{$B'$}{Variable temporaire pour traverser la base de connaissance}
    \item \Arg{$S$}{La liste des symboles initialements vrais}
    \item \Arg{$S'$}{La liste des symboles initialement vrais ainsi que ceux déduits des règles d'induction}
    \item \Arg{\textit{ajouté}}{Si oui ou non un symbole à été rajouté à $S'$ dans le dernier passage}
  \end{itemize}
}
\Data{\Arg{$B$}{BC}; \Arg{$S$}{Liste(Symbole)}}
\Result{\Arg{$S'$}{Liste(Symbole)}}
\Assertion{$\forall R \in B, R \neq$ \textit{règle\_vide}}
\Class{n^2 \cdot q \cdot p}

\Begin({MoteurInférence($B$, $S$)}){
  \Assign{$S'$}{$S$}

  \tcp{\underline{Boucle principale}: Répétée jusqu'à ce qu'il n'y aie plus de modification à $S'$ possibles}
  \Repeat{\textit{ajouté}}{
    \Assign{\textit{ajouté}}{Faux} \\
    \Assign{$B'$}{$B$}

    \tcp{Pour toutes les règles de la base de connaissance...}
    \While{$B' \neq$ \textit{base\_vide}}{
      \tcp{Si la conclusion n'appartient pas à $S'$...}
      \If{\Not{\hyperref[LC]{ListeContient}($S'$, \hyperref[CR]{ConclusionRègle}(\hyperref[TB]{TêteBase}($B'$)))}}{
        \tcp{Si la prémisse est vraie, alors on ajoute la conclusion à $S'$}
        \If{\hyperref[PVr]{PrémisseVraie}(\hyperref[TB]{TêteBase}($B'$), $S'$)}{
          \Assign{$S'$}{\textit{insérer\_tête}($S'$, \hyperref[CR]{ConclusionRègle}(\hyperref[TB]{TêteBase}($B'$)))} \\
          \Assign{\textit{ajouté}}{Vrai}
        }
      }
      \Assign{$B'$}{\textit{suivant}($B'$)}
    }
  }
}

\caption{MoteurInférence\label{MI}}
\end{algorithm}

\begin{algorithm}
\Vars{
  \begin{itemize}
    \item \Arg{$T$}{Un type d'objets comparables}
    \item \Arg{$L$}{Une liste d'objets de type $T$}
    \item \Arg{$E$}{La valeur à trouver dans $L$}
    \item \Arg{\textit{résultat}}{Si oui ou non $E$ est trouvé dans $L$}
  \end{itemize}
}
\Data{\Arg{$L$}{Liste($T$)}; \Arg{$E$}{$T$}}
\Result{\Arg{\textit{résultat}}{Booléen}}
\Class{longueur(L)}

\Begin({ListeContient($L$, $E$)}){
  \uIf{\textit{est\_vide}($L$)}{
    \Assign{\textit{résultat}}{Faux}
  }
  \uElseIf{\textit{tête}($L$) = $E$}{
    \Assign{\textit{résultat}}{Vrai}
  }
  \Else{
    \Assign{\textit{résultat}}{ListeContient(\textit{reste}($L$), $E$)}
  }
}

\caption{ListeContient\label{LC}}
\end{algorithm}


\begin{algorithm}
\Vars{
  \begin{itemize}
  \item \Arg{$S$}{La liste de symboles vrais, de longueur maximale $q$}
  \item \Arg{$R$}{La règle à vérifier, de longueur maximale $p$}
  \item \Arg{$R'$}{Variable utilisée pour traverser $R$}
  \item \Arg{\textit{résultat}}{Si oui ou non la prémisse de la règle est vraie}
  \end{itemize}
}
\Data{\Arg{$R$}{Règle}; \Arg{$S$}{Liste(Symbole)}}
\Result{\Arg{\textit{résultat}}{Booléen}}
\Assertion{$R \neq$ \textit{règle\_vide}}
\Class{p \cdot q}

\Begin({PrémisseVraie($R$, $S$)}){
  \Assign{\textit{résultat}}{Vrai}

  \If{\Not{\hyperref[PV]{PrémisseVide}($R$)}}{
    \Assign{$R'$}{$R$}

    \While{\Et{\textit{suivant}($R'$) $\neq$ \textit{règle\_vide}}{\textit{résultat}}}{
      \If{\Not{\hyperref[LC]{ListeContient}($S$, \hyperref[TR]{TêteRègle}($R'$))}}{
        \Assign{\textit{résultat}}{Faux}
      }

      \Assign{$R'$}{\textit{suivant}($R'$)}
    }
  }
}

\caption{PrémisseVraie\label{PVr}}
\end{algorithm}

\section{Jeux d'essais}

Voici quelques jeux d'essais pour tester les capacités du moteur d'inférence.
Ceux-cis peuvent être retrouvés dans le fichier \textbf{src/test.c} (utilisant les fonctions décrites jusqu'ici).

\end{document}
